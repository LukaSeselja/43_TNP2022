\documentclass[a4paper]{article}

\usepackage{color}
\usepackage{url}
\usepackage[utf8]{inputenc}
\usepackage{graphicx}

\usepackage[english,serbian]{babel}

\usepackage[unicode]{hyperref}
\hypersetup{colorlinks,citecolor=green,filecolor=green,linkcolor=blue,urlcolor=blue}


\begin{document}
\title{VAR tehnologija\\ \small{Seminarski rad u okviru kursa\\Tehničko i naučno pisanje\\ Matematički fakultet}}

\author{Luka Šešelja\\ luka.seselja123@gmail.com \and Ognjen Arsenijević\\ ogiarsenijevic1@gmail.com \and Ognjen Radivojević\\ email}
\date{20.~novembar 2022.}
\maketitle

\begin{abstract}
    
\end{abstract}

\tableofcontents

\newpage

\section{Uvod}
\textbf{Sistem VAR} ili samo \textbf{VAR} (eng.~{\em video assistant referee}) je sistem koji se koristi u fudbalu pri raznim nejasnim situacijama radi proveravanja odluka glavnog sudije pomoću pregleda video snimaka.

Međunarodni odbor fudbalskih saveza (eng.~{\em International Football Association Board (IFAB)}) zvanično je uneo VAR u pravila fudbala 2018. godine, nakon što je isti doživeo probu u brojnim takmičenjima.

Treba napomenuti da se VAR ne odnosi samo na tehnologiju, već i na sudije i pomoćnike koji koriste istu.

\section{Istorija}
\subsection{Kako je VAR nastao?}
VAR je osmišljen u okviru projekta "Refereeing 2.0" početkom 2010-ih, pod upravom Kraljevskog fudbalskog saveza Holandije (eng.~{\em The Royal Netherlands Football Association (KNVB)}). Sistem je testiran tokom sezone 2012-13. godine u okviru Eredivizije (hol.~{\emph{Eredivisie}, u prevodu \emph{Divizija časti} ili \emph{Premijer liga}}), najviše profesionalne fudbalske lige u Holandiji. 

KNVB je 2014. godine poslao peticiju IFAB-u da izmeni pravila fudbala kako bi omogućili probnu upotrebu sistema, što su oni i učinili tokom njihovih redovnih sastanaka 2016. godine.

Probe VAR-a uživo počele su u avgustu 2016. godine, u okviru Druge lige SAD-a, na utakmici rezervnih ekipa dva MLS (skr. od \emph{Major League Soccer}) tima. Glavni sudija, Ismail Elfat proverio je dva faula na utakmici i odlučio je, nakon konsultacija sa VAR sudijom Alenom Čapmanom, da dodeli crveni i žuti karton za prekršaje.

Prva profesionalna utakmica bila je zvanična utakmica prvog kola KNVB kupa 2016. godine. Ovaj meč je bio prvi meč koji je uključivao „monitor pored terena“. Monitor na terenu bi omogućio sudiji da pregleda snimke sa terena. Na osnovu VAR-a, ali bez korišćenja dostupnog monitora na terenu, žuti karton je pretvoren u crveni i tako je ovo bilo prvo isključenje zasnovano na VAR-u u profesionalnoj igri.



\end{document}
