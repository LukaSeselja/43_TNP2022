\documentclass[a4paper]{article}

\usepackage{color}
\usepackage{url}
\usepackage[utf8]{inputenc}
\usepackage{graphicx}

\usepackage[english,serbian]{babel}

\usepackage[unicode]{hyperref}
\hypersetup{colorlinks,citecolor=green,filecolor=green,linkcolor=blue,urlcolor=blue}


\begin{document}
\title{VAR tehnologija\\ \small{Seminarski rad u okviru kursa\\Tehničko i naučno pisanje\\ Matematički fakultet}}

\author{Luka Šešelja\\ luka.seselja123@gmail.com \and Ognjen Arsenijević\\ ogiarsenijevic1@gmail.com \and Ognjen Radivojević\\ email}
\date{20.~novembar 2022.}
\maketitle

\begin{abstract}
    
\end{abstract}

\tableofcontents

\newpage

\section{Uvod}
\textbf{Sistem VAR} ili samo \textbf{VAR} (eng.~{\em video assistant referee}) je sistem koji se koristi u fudbalu pri raznim nejasnim situacijama radi proveravanja odluka glavnog sudije pomoću pregleda video snimaka.

Međunarodni odbor fudbalskih saveza (eng.~{\em International Football Association Board (IFAB)}) zvanično je uneo VAR u pravila fudbala 2018. godine, nakon što je isti doživeo probu u brojnim takmičenjima.

Treba napomenuti da se VAR ne odnosi samo na tehnologiju, već i na sudije i pomoćnike koji koriste istu.

\section{Istorija}
VAR je osmišljen u okviru projekta "Refereeing 2.0" početkom 2010-ih, pod upravom Kraljevskog fudbalskog saveza Holandije (eng.~{\em The Royal Netherlands Football Association (KNVB)}). Sistem je testiran tokom sezone 2012-13. godine u okviru Eredivizije (hol.~{\emph{Eredivisie}, u prevodu \emph{Divizija časti} ili \emph{Premijer liga}}), najviše profesionalne fudbalske lige u Holandiji. 

KNVB je 2014. godine poslao peticiju IFAB-u da izmeni pravila fudbala kako bi omogućili probnu upotrebu sistema, što su oni i učinili tokom njihovih redovnih sastanaka 2016. godine.

Probe VAR-a uživo počele su u avgustu 2016. godine, u okviru Druge lige SAD-a, na utakmici rezervnih ekipa dva MLS (skr. od \emph{Major League Soccer}) tima. Glavni sudija, Ismail Elfat proverio je dva faula na utakmici i odlučio je, nakon konsultacija sa VAR sudijom Alenom Čapmanom, da dodeli crveni i žuti karton za prekršaje.

Prva profesionalna utakmica bila je zvanična utakmica prvog kola KNVB kupa 2016. godine. Ovaj meč je bio prvi meč koji je uključivao „monitor pored terena“. Monitor na terenu bi omogućio sudiji da pregleda snimke sa terena. Na osnovu VAR-a, ali bez korišćenja dostupnog monitora na terenu, žuti karton je pretvoren u crveni i tako je ovo bilo prvo isključenje zasnovano na VAR-u u profesionalnoj igri. Monitor pored terena je sledeći put korišćen na Svetskom klupskom prvenstvu u fudbalu 2016. godine.

A-liga Australije je prva liga u kojoj je korišćen VAR sistem 2017. godine. Korišćen je na utakmici između Melburn Sitija i Adelejd Junajteda. Ipak, u utakmici VAR nije upotrijebljen. Prva intervencija VAR-a u ligaškim utakmicama bila je 8. aprila 2017, na utakmici između Velingtona i Sidneja. VAR je identifikovao nedozvoljeno igranje rukom u šesnaestercu i dosuđen je penal za Sidnej. Utakmica je završena 1:1.

MLS liga uvela je VAR tokom sezone 2017, nakon All-Star utakmice održane 2. avgusta 2017. Prva zvanična upotreba VAR-a bila je na utakmici između Filadelfije i Dalasa, gde je poništen gol zbog kontakta između fudbalera Dalasa i golmana Filadelfije. VAR je takođe korišćen na Kupu konfederacija 2017, gdje je njegova korisnost dovedena u pitanje nakon finala.

Početkom sezone 2017/18, VAR je uveden u mnoge najveće svetske lige, sa izuzetkom Premijer lige. Korišćen je takođe i na Svetskom prvenstvu za igrače do 20 godina 2017. Na dan 8. januara 2018. godine, VAR je po prvi put korišćen u Engleskoj, na utakmici FA Kupa između Brajtona i Kristal Palasa, a narednog dana je upotrijebljen u Liga kupu Francuske, u derbiju Azurne obale između Monaka i Nice.

Na dan 3. marta 2018. godine, IFAB je uneo VAR u pravila fudbala. Pravilo nije obavezno za takmičenja i nije se očekivalo da se odmah, od sezone 2018/19 primeni u Premijer ligi i Ligi šampiona. 

Na Svetskom prvenstvu, VAR je prvi put upotrebljen na prvenstvu 2018. Prvi penal dosuđen nakon konsultacije sa VAR sudijom bio je na utakmici između Francuske i Australije, 16. juna. Antoan Grizman je dao gol. Glavni sudija bio je Andres Kunja, dok je VAR sudija bio Mauro Viljano. Na utakmici između Srbije i Švajcarske, VAR sudija Feliks Cvajer sugerisao je glavnom sudiji — Feliksu Brihu, da je napravljen prekršaj nad Aleksandrom Mitrovićem u šesnaestercu. Brih je odbio sugestiju i nije išao da gleda snimak. Kontakt Mitrovića i dva fudbalera Švajcarske mnogi stručnjaci su okarakterisali kao penal i postavili pitanje čemu služi VAR.

\section{Primena VAR tehnlogije u praksi}
\subsection{Slucajevi u kojima se koristi VAR tehnologija}
Pravila igre ostavljaju prostor za tumačenje i istraživači su pokazali da na sudije mogu uticati spoljašnji faktori kao što su buka publike(Nevill et al.,2002), prednost domaćeg terena(Unkelbach \& Memmert, 2010), kao i prethodne odluke(Plessner \& Betsch, 2001). Odluke koje sudije donose stoga nisu 100\% ispravne, te je VAR osmišljen i uveden kako bi vršio korekciju jasnih i očiglednih grešaka u četiri slučaja:
\begin{itemize}
\item Gola ili prekršaja koji dovodi do gola
\item Penala ili prekršaja koji dovodi do penala
\item Dodele direktnog crvenog kartona
\item Greške pri identifikovanju igrača
\end{itemize}

VAR može da inteveniše kada je pogrešna odluka rezultat jednog ili više koraka u procesu donošenja odluke(Plessner \& Haar, 2006):

Pažnja i percepcija (npr. kada je promašen ozbiljan incident u kaznenom prostoru ili sudija nije video da li je lopta prešla gol liniju ili ne);

Obrada informacija i kategorizacija (npr. kada je ozbiljna faul igra pogrešno kategorisana kao žuti umesto kao crveni karton);

Odgovor na ponašanje (npr. kada je incident dobro shvaćen i kategorisan, ali je žuti karton dat pogrešnom igraču, što je označeno kao pogrešan identitet).
\subsection{Podešavanje kamere}
Sudijski tim video pomoćnika ima pristup četrdeset dve kamere za emitovanje, od kojih su osam super usporene (eng.~{\em super slow motion}) i četiri ultra usporene (eng.~{\em ultra slow motion}). Usporene reprize se uglavnom koriste za činjenične situacije, na primer, da bi se identifikovala tačka kontakta fizičkog prekršaja ili pozicija prekršaja. Ponavljanja normalne brzine se koriste za subjektivne procene, na primer, da bi se utvrdio intenzitet prekršaja ili da li je igra rukom kažnjiva. Pored kamera za emitovanje, VAR tim ima pristup fidovima kamere koje koristi polu-automatska ofsajd tehnologija (eng.~{\em semi-automated offside technology}).


\begin{figure}[h!]
\begin{center}
\includegraphics[scale=0.50]{kamere.jpg.jpg}
\end{center}
\caption{Kamere}
\label{kamere.jpg.jpg}
\end{figure}

\section{Preciznost provera i statistika} \\

Preciznost odluka \\

Ispravnost sudijske odluke podeljena je na dva dela: tačnost početne odluke(odluke pre VAR provere) i tačnost konačne odluke(odluke posle VAR provere). Preciznost sudijskih odluka se meri u procentima i predstavlja procentualni odnos slaganja sudijskih odluka sa odlukama nacionalnog sudijskog komiteta. \\

Trajanje provera i pregleda \\
Trajanje provere se odnosi na vreme potrebno VAR timu da proveri spornu situaciju i utvrdi da li treba da se izvrši pregled. Pregled glavnog sudije na terenu počinje kada rukama pokaže oblik TV ekrana i završava se kada donese konačnu odluku. Prema VAR protokolu, pregled situacije može da se izvrši samo komunikacijom sa VAR sobom, ili se može dopuniti pregledom glavnog sudije na terenu. Trajanje provere u VAR sobi meri se odvojeno od trajanja pregleda na terenu.\\

Statistička analiza \\

Na osnovu ispitanih šansi za tačne početne i konačne odluke koristeći model logističke regresije, u ovoj analizi je zaključeno da svaka situacija jeste doprinela i početnoj i konačnoj (nakon potencijalne intervencije VAR-a) odluci sudije. Odluke nisu bile statistički nezavisne jer su sudije imale više od jedne odluke. Da bi se važio ovaj poslednji faktor, korišćen je model nasumičnih efekata. \\


Preciznost odluka \\

Bilo je ukupno 9732 provere u 2195 utakmica. Sveukupno, 638 situacija je svrstano u sivu zonu, odnosno odluke za koje ne postoji jasna referentna odluka od strane nacionalnog sudijskog komiteta (tj. može se podržati više odluka). Bilo je 99 pregleda za incidente u sivoj zoni, a sudija je promenio prvobitnu odluku 44 puta (7\% svih odluka)
Prvobitna odluka sudije bila je tačna u 8376 od 9094 čiste situacije, što je donelo tačnost odluke od 92,1\%. Nakon intervencije VAR-a, 8942 od 9094 situacije su bile ispravne, što je dalo procenat tačnosti od 98,3\%.  

U 585 od 718 incidenata sa pogrešnim početnim odlukama (81,5\%), odluka je preispitana i 577 ovih odluka je ispravljeno. 

U 111 od 8376 incidenata sa ispravnim početnim odlukama (1,3\%), odluka je preispitana i 11 od njih je preinačeno u netačnu odluku. \\

Srednje trajanje 9732 provere(u proseku 4,4 provere po meču) bilo je 22,0 sekunde. Srednje trajanje svih provera tokom meča bilo je 110,0 sekundi. 
Ukupno, 795 od ovih provera rezultiralo je pregledom (0,36 pregleda po meču u proseku), tačnije 534 su bile provere glavnog sudije na terenu sa srednjim trajanjem od 62,0 sekunde i 261 je bila samo VAR procena u sobi sa srednjim trajanjem od 15,0 sekundi. \\

Bilo je 1544 utakmice bez pregleda(70,3\% svih utakmica); 530 utakmica sa samo jednim pregledom(24,2\% svih mečeva); 103 meča sa 2 pregleda (4,7\% svih mečeva); 15 utakmica sa 3 pregleda (0,7\% svih mečeva); 2 meča sa 4 pregleda (0,1\% svih mečeva) i 1 meč sa 6 pregleda (0,1\% svih mečeva).

Najveći udeo provera imali su incidenti sa crvenim kartonom (39,3\%), zatim kazneni incidenti (33,4\%), golovi (27,1\%) i pogrešan identitet (<1\%). Incidente sa penalima (43,9\%) imali su najveći udeo pregleda, zatim golovi (32,5\%), incidenti sa crvenim kartonom (22,5\%) i pogrešan identitet (1,1\%).

U slučaju pregleda, sudija ima mogućnost da promeni prvobitnu odluku. Bilo je 76 dodatnih penala (164 dosuđenih penala; 88 poništenih), 126 dodatnih crvenih kartona (132 dodeljena crvena kartona; 6 crvenih kartona poništenih) i 114 golova manje (61 gol; 175 poništenih golova) zbog intervencija VAR-a.

\end{document}
