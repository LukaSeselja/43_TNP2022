\documentclass[a4paper]{article}

\usepackage{color}
\usepackage{url}
\usepackage[utf8]{inputenc}
\usepackage{graphicx}

\usepackage[english,serbian]{babel}

\usepackage[unicode]{hyperref}
\hypersetup{colorlinks,citecolor=green,filecolor=green,linkcolor=blue,urlcolor=blue}


\begin{document}
\title{VAR tehnologija\\ \small{Seminarski rad u okviru kursa\\Tehničko i naučno pisanje\\ Matematički fakultet}}

\author{Luka Šešelja\\ luka.seselja123@gmail.com \and Ognjen Arsenijević\\ ogiarsenijevic1@gmail.com \and Ognjen Radivojević\\ email}
\date{20.~novembar 2022.}
\maketitle

\begin{abstract}
    
\end{abstract}

\tableofcontents

\newpage

\section{Uvod}
\textbf{Sistem VAR} ili samo \textbf{VAR} (eng.~{\em video assistant referee}) je sistem koji se koristi u fudbalu pri raznim nejasnim situacijama radi proveravanja odluka glavnog sudije pomoću pregleda video snimaka.

Međunarodni odbor fudbalskih saveza (eng.~{\em International Football Association Board (IFAB)}) zvanično je uneo VAR u pravila fudbala 2018. godine, nakon što je isti doživeo probu u brojnim takmičenjima.

Treba napomenuti da se VAR ne odnosi samo na tehnologiju, već i na sudije i pomoćnike koji koriste istu.

\section{Istorija}
VAR je osmišljen u okviru projekta "Refereeing 2.0" početkom 2010-ih, pod upravom Kraljevskog fudbalskog saveza Holandije (eng.~{\em The Royal Netherlands Football Association (KNVB)}). Sistem je testiran tokom sezone 2012-13. godine u okviru Eredivizije (hol.~{\emph{Eredivisie}, u prevodu \emph{Divizija časti} ili \emph{Premijer liga}}), najviše profesionalne fudbalske lige u Holandiji. 

KNVB je 2014. godine poslao peticiju IFAB-u da izmeni pravila fudbala kako bi omogućili probnu upotrebu sistema, što su oni i učinili tokom njihovih redovnih sastanaka 2016. godine.

Probe VAR-a uživo počele su u avgustu 2016. godine, u okviru Druge lige SAD-a, na utakmici rezervnih ekipa dva MLS (skr. od \emph{Major League Soccer}) tima. Glavni sudija, Ismail Elfat proverio je dva faula na utakmici i odlučio je, nakon konsultacija sa VAR sudijom Alenom Čapmanom, da dodeli crveni i žuti karton za prekršaje.

Prva profesionalna utakmica bila je zvanična utakmica prvog kola KNVB kupa 2016. godine. Ovaj meč je bio prvi meč koji je uključivao „monitor pored terena“. Monitor na terenu bi omogućio sudiji da pregleda snimke sa terena. Na osnovu VAR-a, ali bez korišćenja dostupnog monitora na terenu, žuti karton je pretvoren u crveni i tako je ovo bilo prvo isključenje zasnovano na VAR-u u profesionalnoj igri. Monitor pored terena je sledeći put korišćen na Svetskom klupskom prvenstvu u fudbalu 2016. godine.

A-liga Australije je prva liga u kojoj je korišćen VAR sistem 2017. godine. Korišćen je na utakmici između Melburn Sitija i Adelejd Junajteda. Ipak, u utakmici VAR nije upotrijebljen. Prva intervencija VAR-a u ligaškim utakmicama bila je 8. aprila 2017, na utakmici između Velingtona i Sidneja. VAR je identifikovao nedozvoljeno igranje rukom u šesnaestercu i dosuđen je penal za Sidnej. Utakmica je završena 1:1.

MLS liga uvela je VAR tokom sezone 2017, nakon All-Star utakmice održane 2. avgusta 2017. Prva zvanična upotreba VAR-a bila je na utakmici između Filadelfije i Dalasa, gde je poništen gol zbog kontakta između fudbalera Dalasa i golmana Filadelfije. VAR je takođe korišćen na Kupu konfederacija 2017, gdje je njegova korisnost dovedena u pitanje nakon finala.

Početkom sezone 2017/18, VAR je uveden u mnoge najveće svetske lige, sa izuzetkom Premijer lige. Korišćen je takođe i na Svetskom prvenstvu za igrače do 20 godina 2017. Na dan 8. januara 2018. godine, VAR je po prvi put korišćen u Engleskoj, na utakmici FA Kupa između Brajtona i Kristal Palasa, a narednog dana je upotrijebljen u Liga kupu Francuske, u derbiju Azurne obale između Monaka i Nice.

Na dan 3. marta 2018. godine, IFAB je uneo VAR u pravila fudbala. Pravilo nije obavezno za takmičenja i nije se očekivalo da se odmah, od sezone 2018/19 primeni u Premijer ligi i Ligi šampiona. 

Na Svetskom prvenstvu, VAR je prvi put upotrebljen na prvenstvu 2018. Prvi penal dosuđen nakon konsultacije sa VAR sudijom bio je na utakmici između Francuske i Australije, 16. juna. Antoan Grizman je dao gol. Glavni sudija bio je Andres Kunja, dok je VAR sudija bio Mauro Viljano. Na utakmici između Srbije i Švajcarske, VAR sudija Feliks Cvajer sugerisao je glavnom sudiji — Feliksu Brihu, da je napravljen prekršaj nad Aleksandrom Mitrovićem u šesnaestercu. Brih je odbio sugestiju i nije išao da gleda snimak. Kontakt Mitrovića i dva fudbalera Švajcarske mnogi stručnjaci su okarakterisali kao penal i postavili pitanje čemu služi VAR.



\end{document}
